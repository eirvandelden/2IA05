%%%%%%%%%%%%%%%%%%%%%%%%%%%%%%%%%%%%%%%%%%%%%%%%%%%%%%%%%%
%%
%%  PROJECT: FPAssignments
%%
%%  Created by Etienne van Delden on 09-04-12.
%%  Copyright 2012 Okami Apps. 
%%  
%%  All rights reserved.
%%
%%%%%%%%%%%%%%%%%%%%%%%%%%%%%%%%%%%%%%%%%%%%%%%%%%%%%%%%%%

\documentclass[12pt]{article}

%%%%%%%%%%%%%%%%%%%%%%%%%%%%%%%%%%%%%%%%%%%%%%%%%%%%%%%%%%
%% 
%% MARK: Packages declarations
%%
%%%%%%%%%%%%%%%%%%%%%%%%%%%%%%%%%%%%%%%%%%%%%%%%%%%%%%%%%%

\usepackage{amssymb}
\usepackage{amscd}

\usepackage{verbatim}

%%%%%%%%%%%%%%%%%%%%%%%%%%%%%%%%%%%%%%%%%%%%%%%%%%%%%%%%%%
%% 
%% MARK: Fonts declarations
%%
%%%%%%%%%%%%%%%%%%%%%%%%%%%%%%%%%%%%%%%%%%%%%%%%%%%%%%%%%%

%\usepackage{euler,eucal}

\newcommand\doubleplus{+\kern-1.3ex+\kern0.8ex}


%%%%%%%%%%%%%%%%%%%%%%%%%%%%%%%%%%%%%%%%%%%%%%%%%%%%%%%%%%
%% 
%% Begin Document
%% 
%%%%%%%%%%%%%%%%%%%%%%%%%%%%%%%%%%%%%%%%%%%%%%%%%%%%%%%%%%

\begin{document}
	
	%%%%%%%%%%%%%%%%%%%%%%%%%%%%%%%%%%%%%%%%%%%%%%%%%%%%%%%%%%
	%% 
	%% MARK: Title & Data
	%% 
	%%%%%%%%%%%%%%%%%%%%%%%%%%%%%%%%%%%%%%%%%%%%%%%%%%%%%%%%%%
	
	\title{Assignments for the course Functional Programming}
	
	\author{Etienne van Delden}
		
	\date{\today}
	
	%%%%%%%%%%%%%%%%%%%%%%%%%%%%%%%%%%%%%%%%%%%%%%%%%%%%%%%%%%
	%% 
	%% MARK: Abstract
	%% 
	%%%%%%%%%%%%%%%%%%%%%%%%%%%%%%%%%%%%%%%%%%%%%%%%%%%%%%%%%%
		
	\maketitle
	
	
	%%%%%%%%%%%%%%%%%%%%%%%%%%%%%%%%%%%%%%%%%%%%%%%%%%%%%%%%%%
	%% 
	%% MARK: The sections
	%% 
	%%%%%%%%%%%%%%%%%%%%%%%%%%%%%%%%%%%%%%%%%%%%%%%%%%%%%%%%%%
	
	\section{Programming with lists} 
	
    We consider the datatype $\mathcal{LT}(B)$  of (finite) binary leaf-trees over type $B$ , defined recursively as the smallest of all possible sets satisfying:
    \begin{displaymath}\begin{array}{l}
        \langle b \rangle \in \mathcal{LT}(B) \textrm{, for all }b \in B ,\\
        \langle s,t \rangle \in \mathcal{LT}(B) \textrm{, for all }s,t \in \mathcal{LT}(B).
    \end{array}\end{displaymath}
    
9. Function L, of type $\mathcal{LT}(B) \rightarrow \mathcal{L}_{*}(B)$, is defined by, for all $b \in B$ and $s,t \in \mathcal{LT}(B)$:
    \begin{displaymath}\begin{array}{l}
        L\cdot \langle b \rangle = [b] \\
        L \cdot \langle s,t \rangle = L·s+L·t
    \end{array}\end{displaymath}
    
    \[\ldots \] This inspires us to consider the following generalization of function L , as a function F of type: $\mathcal{L}_{*}(\mathcal{LT}(B)) \rightarrow \mathcal{L}_{*}(B) \rightarrow \mathcal{L}_{*}(B)$, and with this specification, for all $ss \in \mathcal{L}_{*}(\mathcal{LT}(B))$ and $z \in \mathcal{L}_{*}(B)$:
    \begin{displaymath}\begin{array}{l}
        F \cdot ss \cdot z = flt·(L\bullet rev \cdot ss) + z .
    \end{array}\end{displaymath}

(a) Prove that, as stated above, $L\cdot s\doubleplus L\cdot t = flt\cdot (L\bullet[s,t])$.
    \begin{displaymath}\begin{array}{lll}
         & & flt \cdot (L \bullet [s,t]) \\
        = & \{ def \bullet \} & flt \cdot (L\cdot s \doubleplus L\cdot t) \\
        = & \{ def flt \} & flt \cdot (L\cdot s ) \doubleplus  flt \cdot (L\cdot t) \\
        = & \{ def flt \} & L\cdot s \doubleplus  L\cdot t \\
    \end{array}\end{displaymath}



(b) Show that function F indeed is a generalization of L , by showing how L can be defined in terms of F .



(c) Derive an efficient recursive declaration for function F in which L does not occur anymore.
(d) Explain (the usefulness of) the presence of function rev in F ’s specification.    
	
	%%%%%%%%%%%%%%%%%%%%%%%%%%%%%%%%%%%%%%%%%%%%%%%%%%%%%%%%%%
	%% 
	%% MARK: Bibliography
	%% 
	%%%%%%%%%%%%%%%%%%%%%%%%%%%%%%%%%%%%%%%%%%%%%%%%%%%%%%%%%%
	
	\begin{thebibliography}{XXXX}
		
		\bibitem[XXXX]{XXXX}
		\newblock{Name,}
		\newblock{\em Title,}
		\newblock{ vol. X.}
		\newblock{ pp. XX--YY.}
		\newblock{Editor, Town,}
		\newblock{Year.}
		
	\end{thebibliography}
	
\end{document}