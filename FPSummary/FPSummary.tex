%%%%%%%%%%%%%%%%%%%%%%%%%%%%%%%%%%%%%%%%%%%%%%%%%%%%%%%%%%
%%
%%  PROJECT: FPSummary
%%
%%  Created by Etienne van Delden on 04-03-12.
%%  Copyright 2012 Okami Apps. 
%%  
%%  All rights reserved.
%%
%%%%%%%%%%%%%%%%%%%%%%%%%%%%%%%%%%%%%%%%%%%%%%%%%%%%%%%%%%

\documentclass[onesided]{memoir}

%%%%%%%%%%%%%%%%%%%%%%%%%%%%%%%%%%%%%%%%%%%%%%%%%%%%%%%%%%
%% 
%% MARK: Packages declarations
%%
%%%%%%%%%%%%%%%%%%%%%%%%%%%%%%%%%%%%%%%%%%%%%%%%%%%%%%%%%%

\usepackage{/Users/e_vd/Developer/XuBu/XuBu}
%\usepackage{amsthm}



%\theoremstyle{definition} \newtheorem{lemma}{Lemma}

%%%%%%%%%%%%%%%%%%%%%%%%%%%%%%%%%%%%%%%%%%%%%%%%%%%%%%%%%%
%% 
%% Begin Document
%% 
%%%%%%%%%%%%%%%%%%%%%%%%%%%%%%%%%%%%%%%%%%%%%%%%%%%%%%%%%%

\begin{document}
	
	%%%%%%%%%%%%%%%%%%%%%%%%%%%%%%%%%%%%%%%%%%%%%%%%%%%%%%%%%%
	%% 
	%% MARK: Title & Data
	%% 
	%%%%%%%%%%%%%%%%%%%%%%%%%%%%%%%%%%%%%%%%%%%%%%%%%%%%%%%%%%
	
	\title{2IA05\\ 
        Functional \mbox{Programming}}
	
	\author{Etienne van Delden}
		
	\date{\today}
	
	\maketitle
%    \tableofcontents
	
	%%%%%%%%%%%%%%%%%%%%%%%%%%%%%%%%%%%%%%%%%%%%%%%%%%%%%%%%%%
	%% 
	%% MARK: Introduction
	%% 
	%%%%%%%%%%%%%%%%%%%%%%%%%%%%%%%%%%%%%%%%%%%%%%%%%%%%%%%%%%
	
	\chapter*{Introduction} \newpage \pagecolor{white}
    This is a summary for the course 2IA05: Functional Programming on the University of Technology Eindhoven. It was taught during semester 2A 2012.
	
	
	%%%%%%%%%%%%%%%%%%%%%%%%%%%%%%%%%%%%%%%%%%%%%%%%%%%%%%%%%%
	%% 
	%% MARK: The chapters
	%% 
	%%%%%%%%%%%%%%%%%%%%%%%%%%%%%%%%%%%%%%%%%%%%%%%%%%%%%%%%%%
	
    \chapter{General comments and notation} \newpage \pagecolor{white}
    
    \section{Proof by Induction}
    The Induction Hypothesis is assumed during a proof by Induction and does not necessarily have to be written down. 
    
    \section{Finite lists}
    The set $\mathcal{L}_n(A)$ are all (finite) lists, with element type $A$, of length $n,n \geq 0$. Examples/definitions:
    \begin{displaymath}
    \begin{array}{lcl}
        \mathcal{L}_0(A)     & = & \{[]\} \\
        \mathcal{L}_{n+1}(A) & = & \{ a \triangleright s~|~a \in A \land s \in \mathcal{L}_n(A) \} \\
        \mathcal{L}_*(A)     & = & (\biguplus n : n \geq 0: \mathcal{L}_n(A))
    \end{array}
    \end{displaymath}
       
    
    \chapter{Functions} \newpage \pagecolor{white}
    
    \section{``Map'': $\bullet$}
    The function ``map'', denoted as $\bullet$, is the function that applies a function $f$ on each element of a list, with a function $f: A \rightarrow B$. It is specified as:
    \begin{displaymath}
        (f \bullet): \mathcal{L}_*(A) \rightarrow \mathcal{L}_*(B) \qquad 
    \end{displaymath}
        with
    \begin{displaymath}
        (\forall i: 0 \leq i < n: (f\bullet s)\cdot i = f \cdot (s \cdot i))
    \end{displaymath}
    
    for $s \in \mathcal{L}_*(A)$, with $\#s = n$ and $\#(f\bullet s) = \# s $.
    Let's now derive the $\bullet$-function.

    \todo{put the following on the right spot}
        \begin{displaymath}
            f \bullet [] = []
        \end{displaymath}
        The definition follows from the type of the $\bullet$-function.
    

        \subsection{Base}
        For $a \in A, s\in \mathcal{L}_n(A)$, we have the following base case  $i = 0$;
        \begin{displaymath} \begin{array}{llr}
          &  (f\bullet(a\triangleright s)) \cdot 0  & \color{lightgray}{\textrm{specification} \bullet} \\
        = & f \cdot ( (a\triangleright s)\cdot 0)   & \color{lightgray}{\textrm{property~} \triangleright}\\
        = & f \cdot a                               & \color{lightgray}{\triangleright \textrm{-trick}} \\
        = & ( f\cdot a ~\triangleright \textbf{?})\cdot 0 & 
        \end{array} \end{displaymath}

        
        \subsection{Induction Step}
        For $i: 0 \leq i < n$;
        \begin{displaymath} \begin{array}{llr}
        
          &  (f \bullet(a\triangleright s)) \cdot (i+1) & \color{lightgray}{\textrm{specification} \bullet} \\
        = & f \cdot ( (a\triangleright s)\cdot (i+1))   & \color{lightgray}{\textrm{property~} \triangleright}\\
        = & f\cdot (s\cdot i)                            & \color{lightgray}{\textrm{specification $\bullet$, I.H}}\\
        = & (f \bullet s) \cdot i                       & \color{lightgray}{\triangleright \textrm{-trick}}\\
        = & ( \textbf{?~}\triangleright f\bullet s ) \cdot (i+1) & 
        \end{array} \end{displaymath}        

        \subsection{Result}
        To get the final result, we can combine the two derived results and fill in the ``Don't cares'' (\textbf{?}):
        
        \begin{displaymath} \begin{array}{llr}
          &  \textrm{specification ~} f\bullet  & \color{lightgray}{\textrm{Combination}} \\
        \Leftarrow & (\forall i: 0 \leq i < n+1:  & \\
        & \quad (f\bullet(a\triangleright s))\cdot i= & \\
        & \quad (f\cdot a \triangleright f \bullet s)\cdot i ) & \color{lightgray}{\textrm{Leibniz (extensionality)}} \\
        \Leftarrow & f \bullet (a \triangleright s) = (f \cdot a) \triangleright f \bullet s
        \end{array} \end{displaymath}

        Thus, the definition of ``$\bullet$'' is:
        \begin{displaymath} \begin{array}{lcl}
            f \bullet [] & = & [] \\
            f \bullet (a\triangleright s) & = & f\cdot a \triangleright f \bullet s
        \end{array} \end{displaymath}
        
    
            \subsubsection{Lemma's Used}
            \begin{displaymath}
            \begin{array}{lcl}
                (a \triangleright s) \cdot 0 & = &  a \\
                (a\triangleright s)\cdot (i+1) & = & s\cdot i, i \geq 0
            \end{array}
            \end{displaymath}

    \section{``Reverse'': rev}\label{function:rev}
        The function $rev$ reverses the order of all elements in a list.
    The type of $rev$ is
    
    \begin{displaymath}
        \mathcal{L}_n(A) \rightarrow \mathcal{L}_n(a) \qquad n \geq 0
    \end{displaymath}
    with specification
    
    \begin{displaymath}
        (\forall i: 0 \leq i \leq n: rev\cdot s \cdot i = s\cdot (n-i))
    \end{displaymath}
    for all $s \in \mathcal{L}_{n+1}(A)$.
    
    We can now derive the definition of $rev$, by using Induction.
        From the type of $rev$ it follows that:
        \begin{displaymath}
            rev\cdot [] = []
        \end{displaymath}
        
        \subsection{Base}
        
        \begin{displaymath} \begin{array}{llr}
          & rev\cdot(a\triangleright s)\cdot n & \color{lightgray}{\textrm{specification~} rev} \\
        = & (a\triangleright s)\cdot 0  & \color{lightgray}{\textrm{property~} \triangleright}\\
        = & a & \color{lightgray}{\triangleright \textrm{-trick}}\\
        = & [a]\cdot 0 & \color{lightgray}{\textrm{property~} \doubleplus \textrm{; assume} \#t=n}\\
        = & (t \# [a])\cdot n & 
        \end{array} \end{displaymath}        

        \colorbox{lightgray}{\color{white}{As we start counting at 0, this is correct. }}

        \subsection{Step}
        For $s\in\mathcal{L}_N(A)$ and $a\in A$.
        
        \begin{displaymath} \begin{array}{llr}       
          & rev\cdot(a\triangleright s)\cdot i & \color{lightgray}{\textrm{specification~} rev} \\
        = & (a\triangleright s)\cdot (n-i) & \color{lightgray}{\textrm{assume~} i<n, \textrm{property~} \triangleright}\\
        = & s \cdot (n-1-i) & \color{lightgray}{\textrm{Ind. Hypothesis, specification~} rev}\\
        = & rev \cdot s \cdot i & \color{lightgray}{textrm{property~}\doubleplus}\\
        = & (rev \cdot s \doubleplus \boldmath{?}) \cdot i & 
        \end{array} \end{displaymath}

        \subsection{Result}
        We can now combine the results from the Complete Induction derivation to form the definition of $rev$ :
        \begin{displaymath}\begin{array}{lll}
            rev \cdot [] & = & [] \\
            rev \cdot (a \triangleright s) & = & rev \cdot s \doubleplus [a] \\
            rev \cdot s & = & s \cdot n \triangleright rev \cdot (s \lceil n)
        \end{array}\end{displaymath}
        
        This solution runs in $O(n^2)$ time complexity. This can be improved by using a general function that has an accumalator list in which the answer is built. See Section~\label{general:grev}

    \chapter{Generalisation by Abstraction} \newpage \pagecolor{white}
        
        \section{``Reverse'' $grev$} \label{general:grev}
        The function $grev$ is a generalized version of function $rev$ (as defined in Section~\label{function:rev}). It uses an extra parameter $t \in \mathcal{L}_n(A)$ in which the result list is built.
        We specify $grev$ as follows:
        \begin{displaymath}\begin{array}{lll}
            grev \cdot t \cdot s & = & rev \cdot \doubleplus t
        \end{array}\end{displaymath}
        
        \Bold{Use:}
        \begin{displaymath}\begin{array}{lll}
            rev \cdot s & = & grev \cdot [] \cdot s
        \end{array}\end{displaymath}
        
        \subsection{Base case}
        \begin{displaymath}\begin{array}{lll}
              & grev \cdot t \cdot [] & \color{lightgray}{\textrm{specification~} grev} \\
            = & rev \cdot [] \doubleplus t & \color{lightgray}{\textrm{property~} rev}\\
            = & [] \doubleplus t & \color{lightgray}{\textrm{property~} \doubleplus}\\
            = & t & 
        \end{array}\end{displaymath}
        
        \subsection{Induction step}
        \begin{displaymath}\begin{array}{lll}
          & grev \cdot t \cdot (a \triangleright s) & \color{lightgray}{\textrm{specification~} grev} \\
        = & rev \cdot (a \triangleright s ) \doubleplus t & \color{lightgray}{\textrm{property~} rev}\\
        = & & 
        \end{array}\end{displaymath}
        
    \chapter{Divide and Conquer} \newpage \pagecolor{white}
\end{document}